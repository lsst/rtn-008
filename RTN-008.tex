\documentclass[DM,authoryear,toc]{lsstdoc}
% lsstdoc documentation: https://lsst-texmf.lsst.io/lsstdoc.html
\input{meta}

% Package imports go here.

% Local commands go here.

%If you want glossaries
%\input{aglossary.tex}
%\makeglossaries

\title{LSST Processing of Gravitational Wave TOO Data in the Early Operations Era}

% Optional subtitle
% \setDocSubtitle{A subtitle}

\author{%
Leanne Guy
}

\setDocRef{RTN-008}
\setDocUpstreamLocation{\url{https://github.com/rubin-observatory/rtn-008}}

\date{\vcsDate}

% Optional: name of the document's curator
% \setDocCurator{The Curator of this Document}

\setDocAbstract{%
Since the watershed discovery of an electromagnetic counterpart to the LIGO gravitational wave source GW170817, multi-messenger astrophysics has emerged as a major area of strategic focus for the NSF. LSST’s depth, survey speed, and data management systems will make it a key asset in the search for EM counterparts. Exploiting this capability in the first year of LSST operations (GW observing run O4) may require special actions, however. We discuss potential approaches to data access, reference building, and special data processing.
}

% Change history defined here.
% Order: oldest first.
% Fields: VERSION, DATE, DESCRIPTION, OWNER NAME.
% See LPM-51 for version number policy.
\setDocChangeRecord{%
  \addtohist{1}{YYYY-MM-DD}{Unreleased.}{Leanne Guy}
}


\begin{document}

% Create the title page.
\maketitle
% Frequently for a technote we do not want a title page  uncomment this to remove the title page and changelog.
% use \mkshorttitle to remove the extra pages

% ADD CONTENT HERE
% You can also use the \input command to include several content files.

\appendix
% Include all the relevant bib files.
% https://lsst-texmf.lsst.io/lsstdoc.html#bibliographies
\section{References} \label{sec:bib}
\renewcommand{\refname}{} % Suppress default Bibliography section
\bibliography{local,lsst,lsst-dm,refs_ads,refs,books}

% Make sure lsst-texmf/bin/generateAcronyms.py is in your path
\section{Acronyms} \label{sec:acronyms}
\addtocounter{table}{-1}
\begin{longtable}{p{0.145\textwidth}p{0.8\textwidth}}\hline
\textbf{Acronym} & \textbf{Description}  \\\hline

 &  \\\hline
AGN & active galactic nuclei \\\hline
DM & Data Management \\\hline
DM-SST & DM System Science Team \\\hline
DMTN & DM Technical Note \\\hline
DR1 & Data Release 1 \\\hline
GCN & GRB Coordinates Network \\\hline
GW & Gravitational Wave \\\hline
LDO & LSST Document Operations (Document Handle) \\\hline
LIGO & Laser Interferometer Gravitational-Wave Observatory \\\hline
LSE & LSST Systems Engineering (Document Handle) \\\hline
LSR & LSST System Requirements; LSE-29 \\\hline
LSST & Legacy Survey of Space and Time (formerly Large Synoptic Survey Telescope) \\\hline
NSF & National Science Foundation \\\hline
OPS & Operations \\\hline
QA & Quality Assurance \\\hline
RTN & Rubin Technical Note \\\hline
SST & Subsystem Science Team \\\hline
\end{longtable}

% If you want glossary uncomment below -- comment out the two lines above
%\printglossaries





\end{document}
